\section{Conclusions}
\begin{frame}{}
    \tableofcontents[currentsection]
\end{frame}

% \begin{frame}{Conclusions}
%     \begin{itemize}
%         \item A new classification of buses has been proposed. A modified Newton-Raphson algorithm that considers the current equation of converters has been developed.
%         \item It has been found possible in a simple system to algebraically include the current limits of converters with a formulation based on a Lagrangian.
%         \item The voltage-support provided by conventional grid codes can be improved. The adapted grid codes are concluded to be near-optimal.
%         \item The Galerkin method has been studied. A generalized program to solve realistic grids with small errors has been written. 
%         \item The Principal Component Analysis (PCA) combined with dimension reduction yields satisfactory errors and is much more efficient than the Galerkin method. The parametrized states can be used in optimization problems.
%     \end{itemize}
% \end{frame}

\begin{frame}{Conclusions}
    \textcolor{green}{\ding{51}} To analyze voltage spread of distribution grid with high penetration of solar.

    \textcolor{green}{\ding{51}} To develop a dynamic model of the UPQC, including all required controllers.

    \textcolor{green}{\ding{51}} To develop an equivalent static model of the UPQC and validate its behaviour with the dynamic model.

    \textcolor{green}{\ding{51}} To propose design methodology for the converter sizing making use of the developed tools.

    \textcolor{green}{\ding{51}} To show behaviour of the converter over sample days using different design criteria.
\end{frame}



\begin{frame}{Next Steps}
    \begin{itemize}
        \item Find optimal shunt converter current calculation
        \item Extend static model to state space to provide pseudo dynamic behaviour
        \item Create a voltage limit condition dependent on the load
    \end{itemize}
\end{frame}

% \begin{frame}{Future work}
%     \begin{itemize}
%         \item To develop heuristics to determine the states of the converters and how to switch between them in order to get to a solution.
%         \item To develop the Lagrangian formulation for AC systems in order to realistically include converters' current limits.
%         \item Study the optimality of grid codes in distribution grids.
%         \item Parametrize the fault impedances instead of the powers. This way, obtaining results for a sweep of fault impedances could be significantly faster.
%         \item Test other families of bases for the Galerkin method such as Chebyshev polynomials.
%     \end{itemize}
% \end{frame}



